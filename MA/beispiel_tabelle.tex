\documentclass[oneside,12pt,a4paper]{scrreprt}

\usepackage{microtype}

\usepackage{tabularray, xcolor}
\UseTblrLibrary{booktabs}

\usepackage{polyglossia}
\setmainlanguage{english}
\setotherlanguage{german}

\begin{document}

\chapter{Tabellen}
\section{Standart-\LaTeX-Tabelle \texttt{tabular}}

\begin{table}[h!]
    \centering
    \caption{Beispieltabelle mit \texttt{tabular}}
    \begin{tabular}{lcrp{3cm}}
        \toprule
        linksbündig & zentriert & rechtsbündig & umgebrochen \\
        \midrule
        Zelle 1     & Zelle 2   & dritte Zelle & in dieser Zelle wird der Text nach 3\,cm umgebrochen. \\
        \bottomrule
    \end{tabular}
\end{table}


\section{Tabellen mit \texttt{tabularray}}

\begin{table}[hb]
    \caption{Beispieltabelle mit \texttt{tabularray}}
    \begin{tblr}{X[l]X[c]X[r]X[j]}
        \toprule
        linksbündig & zentriert & rechtsbündig & Blocksatz \\
        \midrule
        In der ersten Spalte wird der Text automatisch umgebrochen und linksbündig dargestellt. &
        In der zweiten Spalte wird der Text automatisch umgebrochen und zentriert dargestellt. &
        In der dritten Spalte wird der Text automatisch umgebrochen und rechtsbündig dargestellt. &
        In der vierten Spalte wird der Text automatisch umgebrochen und im Blocksatz dargestellt. \\
        \bottomrule
    \end{tblr}
\end{table}

\begin{table}[hb]
    \caption{Unterschiedliche Aufteilung der Gesamtbreite, negative Koeffizienten legen nur die \emph{maximale} Breite fest.}
    \centering
    \begin{tblr}{
        width = 0.8\textwidth,
        colspec = {|l|X[1]|X[2]|X[-1]|},
    }
        Alpha   & Beta  & Gamma  & Delta \\
        Epsilon & Zeta  & Eta    & Theta \\
        Iota    & Kappa & Lambda & Mu    \\
    \end{tblr}
\end{table}

\begin{table}[hb]
    \caption{Unterschiedlich formatierte Linien}
    \centering
    \begin{tblr}{
        columns = {
            wd=2cm,
            halign=c
        },
        row{2-3} = {font=\itshape},
        vlines = {solid, magenta}, 
        hlines = {dashed},
    }
        Alpha   & Beta  & Gamma  & Delta \\
        Epsilon & Zeta  & Eta    & Theta \\
        Iota    & Kappa & Lambda & Mu    \\
    \end{tblr}
\end{table}

\begin{table}[hb]
    \caption{Unterschiedlich eingefärbte Zellen}
    \centering
    \begin{tblr}{
        hlines = {white},
        vlines = {white},
        cell{1,6}{odd} = {teal7},
        cell{1,6}{even} = {green7},
        cell{2,4}{1,4} = {red7},
        cell{3,5}{1,4} = {purple7},
        cell{2}{2} = {r=4,c=2}{c,azure7},
    }
        Alpha   & Beta  & Gamma   & Delta   \\
        Epsilon & Zeta  & Eta     & Theta   \\
        Iota    & Kappa & Lambda  & Mu      \\ 
        Nu      & Xi    & Omicron & Pi      \\
        Rho     & Sigma & Tau     & Upsilon \\
        Phi     & Chi   & Psi     & Omega   \\
    \end{tblr}
\end{table}


\begin{table}[hb]
    \caption{Tabelle mit der \texttt{booktabs}-Erweiterung}
    \centering
    \begin{booktabs}{
        colspec = lcccc,
        cell{1}{1} = {r=2}{},
        cell{1}{2,4} = {c=2}{},
    }
        \toprule
            Sample & I &   & II & \\
        \cmidrule[lr]{2-3} \cmidrule[lr]{4-5}
                   & A & B & C & D \\
        \midrule
            S1     & 5 & 6 & 7 & 8 \\
            S2     & 6 & 7 & 8 & 5 \\
            S3     & 7 & 8 & 5 & 6 \\
        \bottomrule
    \end{booktabs}
\end{table}


\end{document}
