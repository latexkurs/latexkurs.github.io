% !TEX program = xelatex
% !TEX encoding = utf8
% !TEX spellcheck = de_DE

\documentclass{scrreprt}

\usepackage{polyglossia}
\setmainlanguage{german}

\usepackage{cleveref}


\usepackage{fontspec}
\setsansfont{Linux Biolinum O}
\setromanfont{Linux Libertine O}

\usepackage{
	blindtext,
	booktabs,
	csquotes,
	fontspec,
}

\usepackage{makeidx}
\makeindex

\usepackage{hyperref}


\begin{document}

	\titlehead{\Large Universität Schlauenheim}
	\subject{Masterarbeit}
	\title{Risikomanagement in Zeiten von Social Media}
	\subtitle{Design interaktiver Apps für Banken und Versicherungen}
	\author{cand.\,stup. Uli Ungenau}
	\date{30. Februar 2016}
	\publishers{Betreut durch Prof.\,Dr.\,rer.\,stup. Naseweis}
	\dedication{Für meine Mama.}

	\maketitle
	
	\begin{abstract}
		\noindent\blindtext
	\end{abstract}
	
	\tableofcontents
	\listoffigures
	\listoftables
	
	
	\blinddocument

\begin{figure}

	\caption{bla	\label{fig:ende}}

\end{figure}
	

	Hier kommen zwei Zitate. Zunächst ein kürzeres (siehe Abschnitt \cref{fig:ende}):
	\begin{quote}
		Beware of bugs in the above code; I have only proved it correct, not tried it. \hfill\textit{Knuth}
	\end{quote}
	Und danach ein Längeres:
	\begin{quotation}
		To summarize: We have seen that computer programming is an art, because it applies accumulated knowledge to the world, because it requires skill and ingenuity, and especially because it produces objects of beauty. A programmer who subconsciously views himself as an artist will enjoy what he does and will do it better. Therefore we can be glad that people who lecture at computer conferences speak of the state of the Art. \cite{knuth74}
	\end{quotation}
	
	Hier steht etwas über Gürteltiere. \cite{autor88} \index{Gürteltier} \blindtext
	
	\begin{figure}
		\centering
		\fbox{Das Bild}
		\caption{Ein Bild}
		\label{fig:dasbild}
	\end{figure}
	
	\begin{table}
		\centering
		\begin{tabular}{cc}
			\toprule
			Spalte 1 & Spalte 2\\
			\midrule
			a & b \\
			c & d
		\end{tabular}
		\caption{Eine Tabelle}
		\label{tab:dietabelle}
	\end{table}
	
	Und Jetzt kommt ein Gedicht:
	\begin{verse}
		Drei Ringe den Elbenkönigen hoch im Licht,\\		
		Sieben den Zwergenherrschern in ihren Hallen aus Stein,\\
		Den Sterblichen, ewig dem Tode verfallen, neun,\\
		Einer dem Dunklen Herrn auf dunklem Thron\\
		Im Lande Mordor, wo die Schatten drohn.
		
		Ein Ring, sie zu knechten, sie alle zu finden,\\
				Ins Dunkel zu treiben und ewig zu binden\\		
		Im Lande Mordor, wo die Schatten drohn. \cite[S.\,7]{tolkien66}
	\end{verse}
	
	
	\appendix
		\printindex
	
		\begin{thebibliography}{9}
			\bibitem{knuth74} \textsc{D. E. Knuth}: \textit{„Computer Programming as an Art“},  Communications of the ACM, Bd. 17, Nr. 12, S. 667–673, 1974.
			\bibitem{autor88} \textsc{A. Autor}: \textit{„Das Buch“}, Der Verlag, Leipzig, 1888.
			\bibitem{tolkien66} \textsc{J. R. R. Tolkien}: \textit{„Der Herr der Ringe“}, Klett-Cotta, Stuttgart, 1978.
		\end{thebibliography}

\end{document}
