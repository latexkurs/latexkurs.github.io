\documentclass[ngerman,t,newif]{scrbook}
\usepackage{xltxtra}
\let\!\relax
\def\coolCode{\catcode`\! 0 \catcode`\\ 12}
\begin{document}
Dieses Beispieldokument zeigt eine Änderung der so genannten
category-codes in \TeX. Solche Sachen verwenden Leute, die zeigen
wollen, wie toll sie mit dem Programm umgehen können. Aber es
bringt oft mehr Probleme als Nutzen, deswegen sollte man es von
vornherein nicht in einem ernsthaften Dokument verwenden.
\coolCode
Nach dem Befehl coolCode kann man nun einen \ schreiben, ohne
\ schreiben? Das ist doch ein oft recht nutzloses Zeichen. Also
lassen wir das und beenden diesen Modus mit dem Befehl:
!catcode`!\ 9
Und jetzt hat der Backslash wieder sein normales Verhalten.
\end{document}
\def\test#1{#1}

