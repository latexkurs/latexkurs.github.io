\documentclass{scrartcl}

% for the math exercise:
\usepackage{amsmath, unicode-math}


% for the tabular exercise:
\usepackage{tabularray}
\UseTblrLibrary{booktabs,siunitx}

\begin{document}


\section{Typesetting the Maxwell Equations}

The Maxwell equations represent the relationship between the electric field $\vec E$ and the magnetic field $\vec B$.

\begin{align*}
	\vec\nabla\cdot \vec E &= \frac{\rho}{\varepsilon_0} &
	\vec \nabla \times \vec E &= -\frac{\partial \vec B}{\partial t} \\
	\vec\nabla\cdot \vec B &= 0 &
	\vec \nabla \times \vec B &= \mu_0 \varepsilon_0 \frac{\partial \vec E}{\partial t}	
\end{align*}

Formula \ref{eq:sum} adds all $a_i$ weighted by $c_i$.

\begin{equation}
\sum_{i=1}^n c_i \cdot a_i \label{eq:sum}
\end{equation}


\subsection{With the package \texttt{unicode-math} you could write equivalently}

The Maxwell equations represent the relationship between the electric field $\vec E$ and the magnetic field $\vec B$.

\begin{align*}
	\vec∇·\vec E &= \frac{ρ}{ε_0} &
	\vec ∇×\vec E &= -\frac{∂\vec B}{∂t} \\
	\vec∇·\vec B &= 0 &
	\vec ∇×\vec B &= \mu_0 ε_0 \frac{∂\vec E}{∂t}	
\end{align*}

Formula \ref{eq:sum2} adds all $a_i$ weighted by $c_i$.

\begin{equation}
Σ_{i=1}^n c_i·a_i \label{eq:sum2}
\end{equation}



\section{Creating a Table}

\begin{table}[h!]
	\caption{Quantity of stock items}
	\begin{tblr}{
			colspec={S[table-format=2.0]XS[table-format=4.0]X[3]},
			row{1}={guard}
		}
		\toprule
		Serial No. & Item & Quantity & Description \\
		\midrule
		1 & pencil & 13 & absolute premium quality, especially sharp, hand painted, grade HB\\
		2 & ballpoint pen & 800 & boring description\\
		3 & … & 5 & … \\
		\bottomrule
	\end{tblr}
\end{table}

\end{document}
