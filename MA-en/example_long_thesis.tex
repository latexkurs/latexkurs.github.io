% !TEX program = lualatex
% !TEX encoding = utf8
% !TEX spellcheck = de_DE

\documentclass{scrreprt}

\usepackage{polyglossia}
\setmainlanguage{english}

\usepackage{fontspec}
\setsansfont{Calibri}
\setromanfont{Cambria}

\usepackage{
	blindtext,
	booktabs,
	csquotes,
	fontspec,
}

\usepackage{makeidx}
\makeindex

\usepackage{hyperref, cleveref}


\begin{document}

	\titlehead{\Large University of Smartville}
	\subject{Master's Thesis}
	\title{Risk Management in the Era of Social Media}
	\subtitle{Design of Interactive Apps for Banks and
	Insurance Companies}
	\author{cand. stup. Ian Imprécis}
	\date{February 30, 2024}
	\publishers{Supervised by Prof. Dr. Smartypants}
	\dedication{For my Mom.}

	\maketitle

	\begin{abstract}
		\noindent\blindtext
	\end{abstract}

	\tableofcontents
	\listoffigures
	\listoftables

	\blinddocument

	\begin{figure}
		\caption{bla \label{fig:end}}
	\end{figure}

	Here are two quotes. First, a shorter one (see Section \cref{fig:end}):
	\begin{quote}
		Beware of bugs in the above code; I have only proved it correct, not tried it. \hfill\textit{Knuth}
	\end{quote}
	And then a longer one:
	\begin{quotation}
		To summarize: We have seen that computer programming is an art, because it applies accumulated knowledge to the world, because it requires skill and ingenuity, and especially because it produces objects of beauty. A programmer who subconsciously views himself as an artist will enjoy what he does and will do it better. Therefore we can be glad that people who lecture at computer conferences speak of the state of the Art. \cite{knuth74}
	\end{quotation}

	Here is something about armadillos. \cite{author88} \index{Armadillo} \blindtext[3]

	\begin{figure}
		\centering
		\fbox{The Image}
		\caption{An Image}
		\label{fig:theimage}
	\end{figure}

	\begin{table}
		\centering
		\begin{tabular}{cc}
			\toprule
			Column 1 & Column 2\
			\midrule
			a & b \
			c & d
		\end{tabular}
		\caption{A Table}
		\label{tab:thetable}
	\end{table}

	And now, a poem:
	\begin{verse}
		Three Rings for the Elven-kings under the sky,\
		Seven for the Dwarf-lords in their halls of stone,\
		Nine for Mortal Men doomed to die,\
		One for the Dark Lord on his dark throne\
		In the Land of Mordor where the Shadows lie.
		
		One Ring to rule them all, One Ring to find them,\\
			One Ring to bring them all and in the darkness bind them\\        
		In the Land of Mordor where the Shadows lie. \cite[P.\,7]{tolkien66}
	\end{verse}

	\appendix
	\printindex

	\begin{thebibliography}{9}
	\bibitem{knuth74} \textsc{D. E. Knuth}: \textit{„Computer Programming as an Art“}, Communications of the ACM, Vol. 17, No. 12, pp. 667–673, 1974.
	\bibitem{author88} \textsc{A. Author}: \textit{„The Book“}, The Publisher, Leipzig, 1888.
	\bibitem{tolkien66} \textsc{J. R. R. Tolkien}: \textit{„The Lord of the Rings“}, Klett-Cotta, Stuttgart, 1978.
	\end{thebibliography}

\end{document}
